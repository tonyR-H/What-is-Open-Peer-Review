\abstract{Open Peer Review is a hot topic. However, at present there is neither a standardized definition of this term nor an agreed schema of its features and implementations, which is highly problematic for discussion of OPR’s potential benefits and drawbacks. This paper seeks to resolve these difficulties by analysing the literature for available definitions of “open peer review” and “open review”, which are then codified against a range of independent OPR traits, in order to build a coherent typology of the many different adaptations to the traditional peer review that have come to be signified by the term OPR.}

